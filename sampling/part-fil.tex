\section{Particle filtering}
\subsection{Sequential importance sampling (SIS)}
    Assume the probabilistic graphical model similar to the one in HMMs, where
    \begin{itemize}
        \item $\vec x_t, \vec x_t \subset \mathcal X^D$ and $\vec y_t, \vec y_t \subset \mathcal Y^D$ are the hidden and observed random variables at time $t, t = 1, \dotsc, T$.
        \item The initial state is characterised by $\vec x_1 \sim \mu(\cdot \mid \vec \theta)$ for some known parameter $\vec \theta \subset \Theta$.
        \item The transitions are characterised by $\vec x_t \mid \vec x_{t - 1} \sim f(\cdot \mid \vec x_{t - 1}; \vec \theta )$.
        \item The emmissions are characterised by $\vec y_t \mid \vec x_t \sim g(\cdot \mid \vec x_t; \vec \theta )$.
    \end{itemize}
    We want to sample from the distribution $p(\vec x_{1:t} \mid \vec y_{1:t}; \vec \theta)$. Assume we can sample from the probability distribution with the pdf of the following form
    \begin{align}
        q(\vec x_{1:t} \mid \vec y_{1:t}; \vec \theta)  &= q(\vec x_t \mid \vec x_{1:t - 1}, \vec y_{1:t}; \vec \theta) q(\vec x_{1:t-1} \mid \vec y_{1:t}; \vec \theta) \\
                                                        &= q(\vec x_t \mid \vec x_{1:t - 1}, \vec y_{1:t}; \vec \theta) q(\vec x_{1:t-1} \mid \vec y_{1:t - 1}; \vec \theta) \\
                                                        &= q(\vec x_t \mid \vec x_{t - 1}, \vec y_t; \vec \theta)
    \end{align}
    If we express the pdf of $p$ for $t = 1, \dotsc, T$ in the form of (for convenience, we drop the conditional dependency on $\vec \theta$):
    \begin{align}
        p(\vec x_{1:t} \mid \vec y_{1:t})   &= \frac{p(\vec y_{1:t} \mid \vec x_{1:t}) p(\vec x_{1:t})}{p(\vec y_{1:t})} \\
                                            &= \frac{p(\vec y_t \mid \vec x_{1:t}, \vec y_{1:t - 1}) p(\vec y_{1:t - 1} \mid \vec x_{1:t}) p(\vec x_{1:t})}{p(\vec y_t \mid \vec y_{1:t - 1}) p(\vec y_{1:t - 1})} \\
                                            &= \frac{p(\vec y_t \mid \vec x_{1:t}, \vec y_{1:t - 1}) p(\vec x_{1:t} \mid \vec y_{1:t - 1})}{p(\vec y_t \mid \vec y_{1:t - 1})} \\
                                            &= \frac{p(\vec y_t \mid \vec x_{1:t}, \vec y_{1:t - 1}) p(\vec x_t \mid \vec x_{1:t - 1}, \vec y_{1:t - 1}) p(\vec x_{1:t - 1} \mid \vec y_{1:t - 1})}{p(\vec y_t \mid \vec y_{1:t - 1})} \\
                                            &= \frac{p(\vec y_t \mid \vec x_t) p(\vec x_t \mid \vec x_{t - 1}) p(\vec x_{1:t - 1} \mid \vec y_{1:t - 1})}{p(\vec y_t \mid \vec y_{1:t - 1})} \\
                                            &\propto p(\vec y_t \mid \vec x_t) p(\vec x_t \mid \vec x_{t - 1}) p(\vec x_{1:t - 1} \mid \vec y_{1:t - 1}) \\
                                            &= g(\vec y_t \mid \vec x_t) f(\vec x_t \mid \vec x_{t - 1}) p(\vec x_{1:t - 1} \mid \vec y_{1:t - 1}) 
    \end{align}
    we can write the weight of the sample $\vec x_{1:t}^{(r)}$ from the proposal $q$ to be
    \begin{align}
        w_t^{(r)}   &\propto \frac{p\left(\vec x_{1:t}^{(r)} \mid \vec y_{1:t}\right)}{q\left(\vec x_{1:t}^{(r)} \mid \vec y_{1:t}\right)} \\
                    &\propto \frac{p\left(\vec y_t \mid \vec x_t^{(r)}\right) p\left(\vec x_t^{(r)} \mid \vec x_{t - 1}^{(r)}\right) p\left(\vec x_{1:t - 1}^{(r)} \mid \vec y_{1:t - 1}\right)}{q\left(\vec x_t^{(r)} \mid \vec x_{t-1}^{(r)}, \vec y_t\right) q\left(\vec x_{1:t-1}^{(r)} \mid \vec y_{1:t - 1}\right)} \\
                    &= w_{t - 1}^{(r)} \frac{p\left(\vec y_t \mid \vec x_t^{(r)}\right) p\left(\vec x_t^{(r)} \mid \vec x_{t - 1}^{(r)}\right)}{q\left(\vec x_t^{(r)} \mid \vec x_{t - 1}^{(r)}, \vec y_t\right)} \\
                    &= w_{t - 1}^{(r)} \frac{g\left(\vec y_t \mid \vec x_t^{(r)}\right) f\left(\vec x_t^{(r)} \mid \vec x_{t - 1}^{(r)}\right)}{q\left(\vec x_t^{(r)} \mid \vec x_{t - 1}^{(r)}, \vec y_t\right)} \label{eq:weights}
    \end{align}
    For $t = 1$
    \begin{align}
        w_1^{(r)}   &\propto \frac{p\left(\vec x_1^{(r)} \mid \vec y_1\right)}{q\left(\vec x_1^{(r)} \mid \vec y_1\right)} \\
                    &\propto \frac{p\left(\vec x_1^{(r)}, \vec y_1\right)}{q\left(\vec x_1^{(r)} \mid \vec y_1\right)} \\
                    &\propto \frac{p\left(\vec y_1 \mid \vec x^{(r)}_1\right) p\left(\vec x^{(r)}_1\right)}{q\left(\vec x_1^{(r)} \mid \vec y_1\right)} \\
                    &= \frac{g\left(\vec y_1 \mid \vec x^{(r)}_1\right) \mu\left(\vec x^{(r)}_1\right)}{q\left(\vec x_1^{(r)} \mid \vec y_1\right)} 
    \end{align}
    Note that second line is proportional to the first line with respect to $p(\vec y_1)$ which is justifiable because the the constant of proportionality cancels out during the normalisation step. The algorithm for SIS is shown in Algorithm~\ref{alg:sis} below.
    % \begin{algorithm}
    % \caption{Sequential importance sampling}\label{alg:sis}
    %     \begin{algorithmic}[1]
    %         \State Initialise weights $\left\{w_0^{(r)} = \frac{1}{R}\right\}$.
    %         \For{$t = 1, \dotsc, T$}
    %             \State Observe $\vec y_t$.
    %             \State Sample $\left\{\vec x_{1:t}^{(r)}\right\}$ from $q\left(\vec x_{1:t} \mid \vec y_{1:t}\right)$.
    %             \State Calculate weights $\left\{w_t^{(r)}\right\}$ according to \eqref{eq:weights}.
    %             \State Calculate normalised weights $\left\{\hat w_t^{(r)} = \frac{w_t^{(r)}}{\sum_{r'} w_t^{(r')}}\right\}$.
    %             \State \Comment{The pmf $\sum_r \hat w_t^{(r)} \delta_{\vec x_{1:t}^{(r)}}(\vec x_{1:t})$ approximates the pdf $p(\vec x_{1:t} \mid \vec y_{1:t})$. Hence we can approximate the pdf $p(\vec x_t \mid \vec y_{1:t})$ by $\sum_r \hat w_t^{(r)} \delta_{\vec x_t^{(r)}}(\vec x_t)$.}
    %         \EndFor
    %     \end{algorithmic}
    % \end{algorithm}

    \begin{algorithm}
    \caption{Sequential importance sampling}\label{alg:sis}
        \begin{algorithmic}[1]
            \State Sample from proposal \Comment Initialisation
                \begin{equation}
                    \vec x_1^{(r)} \sim q\left(\cdot \mid \vec y_1^{(r)}, \vec \theta\right), r = 1, \dotsc, R
                \end{equation}
            \State Compute weights
                \begin{equation}
                    w_1^{(r)} \propto \frac{g\left(\vec y_1 \mid \vec x^{(r)}_1\right) \mu\left(\vec x^{(r)}_1\right)}{q\left(\vec x_1^{(r)} \mid \vec y_1\right)}, r = 1, \dotsc, R
                \end{equation}
            \State Normalise weights
                \begin{equation}
                    \hat w_1^{(r)} = \frac{w_1^{(r)}}{\sum_{r'} w_1^{(r')}}, r = 1, \dotsc, R
                \end{equation}
            \State We can resample from 
                \begin{equation}
                    \hat p(\mathrm d \vec x_1 \mid \vec y_1, \vec \theta) = \sum_r \hat w_1^{(r)} \delta_{\vec x_1^{(r)}}(\mathrm d\vec x_1)
                \end{equation}
                to estimate
                \begin{equation}
                    p(\vec x_1 \mid \vec y_1, \vec \theta)
                \end{equation}
            \For{$t = 2, \dotsc, T$} \Comment Main loop
                \State Compute weights
                    \begin{equation}
                        w_t^{(r)} \propto w_{t - 1}^{(r)} \frac{p\left(\vec y_t \mid \vec x_t^{(r)}, \vec \theta\right) p\left(\vec x_t^{(r)} \mid \vec x_{t - 1}^{(r)}, \vec \theta \right)}{q\left(\vec x_t^{(r)} \mid \vec x_{t - 1}^{(r)}, \vec y_t, \vec \theta \right)}, r = 1, \dotsc, R
                    \end{equation}
                \State Normalise weights
                    \begin{equation}
                        \hat w_t^{(r)} = \frac{w_t^{(r)}}{\sum_{r'} w_t^{(r')}}, r = 1, \dotsc, R
                    \end{equation}
                \State We can resample from
                    \begin{equation}
                        \hat p(\mathrm d \vec x_{1:t} \mid \vec y_{1:t}, \vec \theta) = \sum_r \hat w_t^{(r)} \delta_{\vec x_{1:t}^{(r)}}(\mathrm d \vec x_{1:t})
                    \end{equation}
                    to estimate
                    \begin{equation}
                        p(\vec x_{1:t} \mid \vec y_{1:t}, \vec \theta)
                    \end{equation}
            \EndFor
        \end{algorithmic}
    \end{algorithm}
    The reason why it works is the same as in the case of Sampling importance resampling described in section \ref{section:sir}.

\subsection{The degeneracy problem}
    Because the support of the pdf we are approximating ($p(\vec x_{1:t} \mid \vec y_{1:t})$) is growing, the constant number of weights we use ($R$) won't be sufficient after a while. This is because many weights will become very negligible, wasting our resources. An \textbf{effective sample size} is used to measure this degeneracy is defined to be and approximated by the following:
    \begin{align}
        S_\text{eff}        &\triangleq \frac{S}{1 + \var\left[{w_t^{(r)}}^\ast\right]} \\
        \hat S_\text{eff}   &\approx \frac{1}{\sum_r \left(w_t^{(r)}\right)^2}\label{eq:Seff}
    \end{align}
    where ${w_t^{(r)}}^\ast = p(\vec x_t^{(r)} \mid \vec y_{1:t}) / q(\vec x_t^{(r)} \mid \vec x_{t-1}^{(r)}, \vec y_t)$ is the ``true weight'' of particle $r$.

    There are (among others) two solutions to this problem -- introduce the resampling step, and using a good proposal distribution.

\subsection{The resampling step}
    Whenever the effective sample size drops below some threshold, resample to get new $R$ samples from the approximation of the pdf. This step is also called \textbf{rejuvenation}. The full algorithm for a generic particle filter is shown in Algorithm~\ref{alg:part-fil} below in which we resample during every tie step.
    \begin{algorithm}
    \caption{Generic particle filter}\label{alg:part-fil}
        \begin{algorithmic}[1]
            \State Sample from proposal \Comment Initialisation
                \begin{equation}
                    \vec x_1^{(r)} \sim q\left(\cdot \mid \vec y_1^{(r)}, \vec \theta\right), r = 1, \dotsc, R
                \end{equation}
            \State Compute weights
                \begin{equation}
                    w_1^{(r)} \propto \frac{p\left(\vec x_1^{(r)} \mid \vec y_1, \vec \theta\right)}{q\left(\vec x_1^{(r)} \mid \vec y_1, \vec \theta \right)}, r = 1, \dotsc, R
                \end{equation}
            \State Normalise weights
                \begin{equation}
                    \hat w_1^{(r)} = \frac{w_1^{(r)}}{\sum_{r'} w_1^{(r')}}, r = 1, \dotsc, R
                \end{equation}
            \State We can resample from 
                \begin{equation}
                    \hat p(\mathrm d \vec x_1 \mid \vec y_1, \vec \theta) = \sum_r \hat w_1^{(r)} \delta_{\vec x_1^{(r)}}(\mathrm d\vec x_1)
                \end{equation}
                to estimate
                \begin{equation}
                    p(\vec x_1 \mid \vec y_1, \vec \theta)
                \end{equation}
            \For{$t = 2, \dotsc, T$} \Comment Main loop
                \State Sample parents' indices of $t^{\text{th}}$ generation
                    \begin{equation}
                        A_{t - 1}^{(r)} \sim \Cat\left(\hat w_{t - 1}\right), r = 1, \dotsc, R
                    \end{equation}
                \State Sample $t^{\text{th}}$ generation using corresponding parents
                    \begin{equation}
                        \vec x_t^{(r)} \sim q\left(\cdot \mid \vec x_{t - 1}^{A_{t - 1}^{(r)}}, \vec y_t, \vec \theta\right), r = 1, \dotsc, R
                    \end{equation}
                \State Compute weights
                    \begin{equation}
                        w_t^{(r)} \propto w_{t - 1}^{(r)} \frac{p\left(\vec y_t \mid \vec x_t^{(r)}, \vec \theta\right) p\left(\vec x_t^{(r)} \mid \vec x_{t - 1}^{(r)}, \vec \theta \right)}{q\left(\vec x_t^{(r)} \mid \vec x_{t - 1}^{(r)}, \vec y_t, \vec \theta \right)}, r = 1, \dotsc, R
                    \end{equation}
                \State Normalise weights
                    \begin{equation}
                        \hat w_t^{(r)} = \frac{w_t^{(r)}}{\sum_{r'} w_t^{(r')}}, r = 1, \dotsc, R
                    \end{equation}
                \State We can resample from
                    \begin{equation}
                        \hat p(\mathrm d \vec x_{1:t} \mid \vec y_{1:t}, \vec \theta) = \sum_r \hat w_t^{(r)} \delta_{\vec x_{1:t}^{(r)}}(\mathrm d \vec x_{1:t})
                    \end{equation}
                    to estimate
                    \begin{equation}
                        p(\vec x_{1:t} \mid \vec y_{1:t}, \vec \theta)
                    \end{equation}
            \EndFor

        \end{algorithmic}
    \end{algorithm}

\subsection{Particle filter animation}
\animategraphics[controls, width=1\linewidth]{0.3}{animations/particle-filter/}{1}{36}

\subsection{The proposal distribution}
    It is common to use the following proposal distribution
    \begin{align}
        q\left(\vec x_{1:t}^{(r)} \mid \vec y_{1:t}\right)  &= q\left(\vec x_t^{(r)} \mid \vec x_{t - 1}^{(r)}, \vec y_t\right) \\
                                                            &= p\left(\vec x_t^{(r)} \mid \vec x_{t - 1}^{(r)}\right) \\
                                                            &= f\left(\vec x_t^{(r)} \mid \vec x_{t - 1}^{(r)}\right)
    \end{align}
    Hence the weight equation in \eqref{eq:weights} becomes
    \begin{align}
        w_t^{(r)}   &\propto w_{t - 1}^{(r)} \frac{g\left(\vec y_t \mid \vec x_t^{(r)}\right) f\left(\vec x_t^{(r)} \mid \vec x_{t - 1}^{(r)}\right)}{q\left(\vec x_t^{(r)} \mid \vec x_{t - 1}^{(r)}, \vec y_t\right)} \\
                    &= w_{t - 1}^{(r)}\, g\left(\vec y_t \mid \vec x_t^{(r)}\right)
    \end{align}
    This approach can be \textbf inefficient because the likelihood, $p\left(\vec y_t \mid \vec x_t^{(r)}\right)$, can be very small at many places meaning many of the particles will be very small.

    The optimal proposal distribution has the form
    \begin{align}
        q\left(\vec x_{1:t}^{(r)} \mid \vec y_{1:t}\right)  &= q\left(\vec x_t^{(r)} \mid \vec x_{t - 1}^{(r)}, \vec y_t\right) \\
                                                            &= p\left(\vec x_t^{(r)} \mid \vec x_{t - 1}^{(r)}, \vec y_t\right) \\
                                                            &= \frac{p\left(\vec y_t \mid \vec x_t, \vec x_{t-1}^{(r)}\right) p\left(\vec x_t, \vec x_{t-1}^{(r)}\right)}{p\left(\vec x_{t-1}^{(r)}, \vec y_t\right)} \\
                                                            &= \frac{p\left(\vec y_t \mid \vec x_t\right) p\left(\vec x_t \mid \vec x_{t-1}^{(r)}\right)}{p\left(\vec y_t \mid \vec x_{t-1}^{(r)}\right)} \\
                                                            &= \frac{g\left(\vec y_t \mid \vec x_t\right) f\left(\vec x_t \mid \vec x_{t-1}^{(r)}\right)}{p\left(\vec y_t \mid \vec x_{t-1}^{(r)}\right)}
    \end{align}
    The weight equation in \eqref{eq:weights} becomes
    \begin{align}
        w_t^{(r)}   &\propto w_{t - 1}^{(r)}\, p\left(\vec y_t \mid \vec x_{t-1}^{(r)}\right) \\
                    &= w_{t - 1}^{(r)} \int{p\left(\vec y_t, \vec x'_t \mid \vec x_{t-1}^{(r)}\right) \,\mathrm d\vec x'} \\
                    &= w_{t - 1}^{(r)} \int{p\left(\vec y_t \mid \vec x'_t, \vec x_{t-1}^{(r)}\right) p\left(\vec x'_t \mid \vec x_{t-1}^{(r)}\right) \,\mathrm d\vec x'} \\
                    &= w_{t - 1}^{(r)} \int{p\left(\vec y_t \mid \vec x'_t\right) p\left(\vec x'_t \mid \vec x_{t-1}^{(r)}\right) \,\mathrm d\vec x'} \\
                    &= w_{t - 1}^{(r)} \int{g\left(\vec y_t \mid \vec x'_t\right) f\left(\vec x'_t \mid \vec x_{t-1}^{(r)}\right) \,\mathrm d\vec x'}
    \end{align}
    The proposal distribution is optimal because for any fixed $\vec x_{t-1}^{(r)}$, the new weight $w_t^{(r)}$ takes the same value regardless of the value drawn for $\vec x_t^{(r)}$. Hence, conditional on the old values, the variance of true weights is zero.