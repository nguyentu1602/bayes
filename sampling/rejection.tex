\section{Rejection sampling}

Assume we can sample from a proposal distribution $q$ with a pdf $q(\vec x)$, which can be evaluated within a multiplicative factor (i.e. we can only evaluate $q^\ast(\vec x) = Z_q q(\vec x)$). Also assume we know the value of a constant $c$ such that
\begin{align}
    c q^\ast(\vec x) > p^\ast(\vec x) & \text{ for all } \vec x
\end{align}

The procedure that generates a sample $\vec x \sim p$ is described in Algorithm~\ref{alg:rejection} below.
\begin{algorithm}
\caption{Rejection sampling}\label{alg:rejection}
    \begin{algorithmic}[1]
        \State Generate $\vec x \sim q$.
        \State Generate $u \sim \Unif(0, c q^\ast(\vec x))$.
        \State If $u > p^\ast(\vec x)$ it is rejected, otherwise it is accepted.
    \end{algorithmic}
\end{algorithm}


\subsection{Why it works?}
    Assume $\vec x \in \mathbb R^D$. Define sets $\mathcal X$ and $\mathcal X'$ to be
    \begin{align}
        \mathcal X &= \left\{\vec \alpha \in \mathbb R^{d + 1}: \alpha_{1:d} \in \mathbb R^d, \alpha_{d + 1} \in [0, c q^\ast(\vec \alpha)]\right\} \\
        \mathcal X' &= \left\{\vec \alpha \in \mathbb R^{d + 1}: \alpha_{1:d} \in \mathbb R^d, \alpha_{d + 1} \in [0, p^\ast(\vec \alpha)]\right\}
    \end{align}
    Note that $\mathcal X' \subseteq \mathcal X$.

    By definition, $\mathcal X$ is the support of $(\vec x, u)$. The probability of $(\vec x, u)$ can be expressed as
    \begin{align}
        \Pr(\vec x, u)  &= \Pr(\vec x) \Pr(u) \\
                        &= q(\vec x) \frac{1}{c q^\ast(\vec x)} \\
                        &= q(\vec x) \frac{1}{c Z_q q(\vec x)} \\
                        &= \frac{1}{c Z_q}
    \end{align}
    which is constant w.r.t. $(\vec x, u)$, i.e.
    \begin{equation}
        (\vec x, u) \sim \Unif(\mathcal X)
    \end{equation}

    Let $(\vec x', u')$ be the value of $(\vec x, u)$ that gets accepted. By definition, $\mathcal X'$ is the support of $(\vec x', u')$:
    \begin{equation}
        (\vec x', u') =
            \begin{cases}
                (\vec x, u)     & \text{if } (\vec x, u) \in \mathcal X' \\
                \text{nothing}  & \text{otherwise.}
            \end{cases}
    \end{equation}
    The probability of $(\vec x', u')$ can be expressed as
    \begin{equation}
        \Pr(\vec x', u') =
            \begin{cases}
                \Pr(\vec x, u)  & \text{if } (\vec x, u) \in \mathcal X' \\
                0               & \text{otherwise.}
            \end{cases}
    \end{equation}
    which means
    \begin{equation}
        (\vec x', u') \sim \Unif(\mathcal X')
    \end{equation}

    Working backwards
    \begin{align}
        \Pr(\vec x')    &= \frac{\Pr(\vec x', u')}{\Pr(u')} \\
                        &\propto \frac{1}{1 / p^\ast(\vec x')} \\
                        &\propto p^\ast(\vec x')
    \end{align}
    Hence the accepted $\vec x$, $\vec x'$ is $\sim p$.