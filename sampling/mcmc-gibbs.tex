\subsection{Gibbs sampling}
Assume we want to sample from $p(\vec x) = p(x_1, \dotsc, x_D)$. We can only sample from the conditionals $p(x_i \mid \vec x_{-i})$ where $\vec x_{-i}$ denotes $\vec x$ with the $i^{\text{th}}$ component ommited. The Gibbs sampling algorithm (\ref{alg:gibbs}) is given below.
\begin{algorithmbis}[Gibbs sampling algorithm]\label{alg:gibbs}
    \begin{algorithmic}[1]
        \State Sample $\vec x^{(0)}$ from an arbitrary probability distribution over $\mathcal X$.
        \For{$t = 1, \dotsc, T$}
            \State Sample $x_1^{(t)} \sim p\left(\cdot \mid x_2^{(t - 1)}, x_3^{(t - 1)}, \dotsc, x_D^{(t - 1)}\right)$
            \State Sample $x_2^{(t)} \sim p\left(\cdot \mid x_1^{(t)}, x_3^{(t - 1)}, \dotsc, x_D^{(t - 1)}\right)$
            \State $\vdots$
            \State Sample $x_D^{(t)} \sim p\left(\cdot \mid x_1^{(t)}, x_2^{(t)}, \dotsc, x_{D - 1}^{(t)}\right)$
        \EndFor
    \end{algorithmic}
\end{algorithmbis}

\subsubsection{Why it works?}
Each of the sampling steps can be viewed to be governed by a different kernel with the whole process being governed by the aggregate kernel. We prove that the single kernels follow the DB equation with respect to $p$:
\begin{align}
    p(\vec x) \mathcal T_i(\vec x \to \vec x')  &= p(\vec x) p(\vec x_{-i}, x'_i \mid \vec x) \\
                                                &= p(\vec x_{-i}, x'_i, \vec x) \\
                                                &= p(\vec x, x'_i, \vec x_{-i}) \\
                                                &= p(\vec x') p(\vec x \mid x'_i, \vec x_{-i}) \\
                                                &= p(\vec x') \mathcal T_i (\vec x' \to \vec x)
\end{align}
This is the premise of Proposition~\ref{proposition:seq}, hence the aggregate kernel $\mathcal T$ has $p$ as its stationary distribution.

We can also view Gibbs sampling as an instance of the MH algorithm. If the proposal of MH $q_i(\vec x \to \vec x')$ is set to be $p(\vec x' \mid \vec x) = p(x_i' \mid \vec x)$ the acceptance probability is one (shown below) and so it is equivalent to one sampling step in Gibbs sampling.
\begin{align}
    \mathcal A(\vec x \to \vec x')  &= \min\left(1, \frac{p(\vec x')p(\vec x \mid \vec x')}{p(\vec x)p(\vec x' \mid \vec x)}\right) \\
                                    &= \min\left(1, \frac{p(\vec x', \vec x)}{p(\vec x', \vec x)}\right) \\
                                    &= 1
\end{align}