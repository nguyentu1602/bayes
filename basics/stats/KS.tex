\subsection{Kolmogorov-Smirnov test}
\subsubsection{Kolmogorov-Smirnov statistic}
Null hypothesis, often denoted by $H_0$ is a general statement or a default position saying there is no relationship between two measured phenomena. 

The Kolmogorov (KS) test quantifies a distance between
\begin{itemize}
	\item The empirical distribution function (or the empirical cdf) and the cdf of the reference function ($H_0 = $ sample is drawn from the reference distribution), or
	\item The empirical cdfs of two samples ($H_0 = $ samples are drawn from the same distribution).
\end{itemize}

The empirical cdf $F_N$ for $N$ iid observations $\{x_n\}$ is
\begin{equation}
	F_N(x) \triangleq \frac{1}{N} \sum_n I(x_n \leq x)
\end{equation}
basically $F_N(x) = \frac{1}{N} \text{number of samples less than or equal to } x$.

The KS statistic for a given cdf $F(x)$ is
\begin{equation}
	D_N(x) \triangleq \sup_x \left|F_N(x) - F(x)\right|
\end{equation}
By Glivenko-Cantelli theorem, if $\{x_n\} \sim F$, then $D_N \to 0$ almost surely when $N \to \infty$.