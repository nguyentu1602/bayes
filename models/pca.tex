\section{Principal components analysis}
\subsection{Classical PCA}
We have data points $\left\{\vec x_n, \vec x_n \in \mathbb R^D\right\}, n = 1, \dotsc, N$. The goal is to project to a lower dimensional space with dimension $M, M < D$, while maximising the variance to get data points in the \emph{principal space}, $\left\{\vec z_n, \vec z_n \in \mathbb R^M\right\}, n = 1, \dotsc, N$. Let the \emph{principal components} be $\left\{\vec u_m, \vec u_m \in \mathbb R^D, \| \vec u_m \| = 1\right\}, m = 1, \dotsc, M$. The projected data can be expressed as
\begin{align*}
    \vec z_n    &= 
        \begin{bmatrix}
            \vec u_1^T \vec x_n \\
            \vdots \\
            \vec u_M^T \vec x_n
        \end{bmatrix} \\
                &= \vec U^T \vec x_n
\end{align*}
for $n = 1, \dotsc, N$ where $\vec U = [\vec u_1, \dotsc, \vec u_M]$.

The total variance we are trying to maximise, i.e. the sum of variances along the dimensions $\left\{\vec u_m\right\}$ is
\begin{align}
    V   &= \sum_{m = 1}^M \var(\text{dimension } m) \nonumber\\
        &= \sum_{m = 1}^M \frac{1}{N} \sum_{n = 1}^N \left( z_{nm} - \mean z_m \right)^2 \nonumber\\
        & \left(\text{where } \mean z_m = \frac{1}{N} \sum_{n = 1}^N z_{nm} \right) \label{eqn:models-pca-dataMean} \\
        &= \frac{1}{N} \sum_{m = 1}^M \sum_{n = 1}^N \left( z_{nm}^2 - 2 z_{nm} \bar z_m + \bar z_m^2 \right) \nonumber\\
        &= \frac{1}{N} \sum_{m = 1}^M \sum_{n = 1}^N \left( \left(\vec u_m^T \vec x_n \right)^2 - 2 \left(\vec u_m^T \vec x_n\right) \left(\vec u_m^T \bar{\vec x}\right) + \left( \vec u_m^T \bar{\vec x} \right)^2 \right), \text{where } \bar{\vec x} = \frac{1}{N} \sum_{n = 1}^N \vec x_n \nonumber\\
        &= \sum_{m = 1}^M \vec u_m^T \left( \frac{1}{N} \sum_{n = 1}^N \vec x_n \vec x_n^T - 2 \vec x_n \bar{\vec x}^T + \bar{\vec x} \bar{\vec x}^T \right) \vec u_m \nonumber\\
        &= \sum_{m = 1}^M \vec u_m^T \left( \frac{1}{N} \sum_{n = 1}^N (\vec x_n - \bar{\vec x})(\vec x_n - \bar{\vec x})^T \right) \vec u_m \nonumber\\
        &= \sum_{m = 1}^M \vec u_m^T \vec S \vec u_m \label{eqn:models-pca-totalVariance}\\
        &\left( \text{where } \vec S = \frac{1}{N} \sum_{n = 1}^N (\vec x_n - \bar{\vec x})(\vec x_n - \bar{\vec x})^T \right) \label{eqn:models-pca-dataCov}
\end{align}

We want to maximise this with the constraint $\|\vec u_m\| = 1, m = 1, \dotsc, M$ which is equivalent to $\vec u_m^T \vec u_m = 1, m = 1, \dotsc, M$. We use Lagrange multipliers $\vec \lambda = (\lambda_1, \dotsc, \lambda_M)$. Hence we need to maximise the following Lagrangian
\begin{equation*}
    \mathcal L(\vec \lambda, \vec u_1, \dotsc, \vec u_M) = \sum_{m = 1}^M \vec u_m^T \vec S \vec u_m + \vec \lambda^T
        \begin{bmatrix}
            1 - \vec u_1^T \vec u_1 \\
            \vdots \\
            1 - \vec u_M^T \vec u_M
        \end{bmatrix}
\end{equation*}

We know that $\vec S$ is positive semi-definite because it is a covariance matrix for $\{\vec x_n\}$. The term $\vec u_m^T \vec S \vec u_m$ is convex w.r.t. $\vec u_m$ because the Hessian $2 \vec S$ is positive semi-definite. Hence $\sum_{m = 1}^M \vec u_m^T \vec S \vec u_m$ must be convex w.r.t. $(\vec u_1, \dotsc, \vec u_M)$. Also, the second term in the Lagrangian is convex w.r.t. the principal components. Hence, we can maximise the Lagrangian by setting the gradients to zero:
\begin{align}
    \grad_{\vec \lambda}{\mathcal L}    &= \vec 0 \label{eqn:gradLambda}\\
    \grad_{\vec u_m}{\mathcal L}        &= \vec 0, m = 1, \dotsc, M \label{eqn:gradU}
\end{align}

From \eqref{eqn:gradLambda}, we obtain $\vec u_m^T \vec u_m = 1, m = 1, \dotsc, M$. From \eqref{eqn:gradU}, we obtain
\begin{align}
    \grad_{\vec u_m}{\mathcal L}    &= 2 \vec S \vec u_m - 2 \lambda_m \vec u_m \\
                                    &= 0 \\
    \implies \vec S \vec u_m        &= \lambda_m \vec u_m \label{eqn:pcaEig}
\end{align}
Thus we can see that $\{\vec u_m\}$ should be selected to be the eigenvectors corresponding to the eigenvalues $\{\lambda_m\}$ of $\vec S$. If we premultiply \eqref{eqn:pcaEig} by $\vec u_m^T$, we get $\lambda_m = \vec u_m^T \vec S \vec u_m$ which can be substituted back to total variance
\begin{equation*}
    V = \sum_{m = 1}^M \lambda_m
\end{equation*}
from which we can see that to maximise, we set $\{\lambda_m\}$ to be the largest $M$ eigenvalues of $\vec S$. The principal components $\{\vec u_m\}$ are the corresponding eigenvectors.

\subsection{Probabilistic PCA}
Following the mixture model, where $\vec Z = \left\{\vec z_n, \vec z_n \in \mathbb R^M\right\}$, $n = 1, \dotsc, N$ are the latent variables and $\vec X = \left\{\vec x_n, \vec x_n \in \mathbb R^D\right\}$, $n = 1, \dotsc, N$ are the observed variables, probabilistic PCA assumes $\mathbb R^M$ is the lower-dimensional space we want to project our data in $\mathbb R^D$ to. We have the following assumtions:
\begin{align*}
    p(\vec z)               &= \Gauss(\vec z; \vec 0, \vec I) \\
    p(\vec x \mid \vec z)   &= \Gauss(\vec x; \vec W \vec z + \vec \mu, \sigma^2 \vec I)
\end{align*}
where $\vec 0, \vec I, \vec W, \vec \mu, \vec I$ all have the appropriate dimensions. Following Subsection~\ref{ssec:prob-gaussian-lingauss}, we can express the remaining marginal and conditional as
\begin{align*}
    p(\vec x)               &= \Gauss(\vec x; \vec \mu, \vec C) \\
    p(\vec z \mid \vec x)   &= \Gauss(\vec z; \vec M^{-1} \vec W^T (\vec x - \vec \mu), \sigma^2 \vec M^{-1})
\end{align*}
where
\begin{align*}
    \vec C      &= \vec W \vec W^T + \sigma^2 \vec I \\
    \vec M      &= \vec W^T \vec W + \sigma^2 \vec I
\end{align*}

\subsubsection{MLE for probabilistic PCA}
To find ML estimates for our model, we want to maximise the following likelihood function:
\begin{align*}
    p(\mathcal D \mid \vec \theta)  &= \prod_{n = 1}^N p(\vec x_n \mid \vec \theta) \\
                                    &= \prod_{n = 1}^N \Gauss(\vec x_n ; \vec \mu, \vec C)
\end{align*}

Maximising this w.r.t. the parameters $\vec W$ and $\sigma^2$, we get the following MLEs:
\begin{align*}
    \vec W_{ML}     &= \vec U_M \left(\vec L_M - \sigma^2 \vec I \right)^{1 / 2} \vec R \\
    \sigma^2_{ML}   &= \frac{1}{D - M} \sum_{i = M + 1}^D \lambda_i
\end{align*}
where $\vec R, \vec R \in \mathbb R^{M \times M}, \vec R \vec R^T = \vec I$ is an arbitrary orthogonal matrix and
\begin{align*}
    \vec U_M    &= \left[\vec u_1, \dotsc, \vec u_M \right] \\
    \vec L_M    &= \diag(\lambda_1, \dotsc, \lambda_M)
\end{align*}
where $\vec u_1, \dotsc, \vec u_D$ and $\lambda_1, \dotsc, \lambda_D$ are eigenvectors and eigenvalues of the data covariance matrix $\vec S$ (defined below in \eqref{eqn:models-pca-dataCov}), sorted in descending order.

\subsubsection{Other stuff to note}
\paragraph{Alternative view.} fdsaf a
\paragraph{Intuitive view.} fsda 
\paragraph{Redundancy in parameterisation.} f ds
\paragraph{Computational complexity.}  fsdaf 

\subsubsection{EM algorithm for probabilistic PCA}

\subsubsection{Bayesian PCA}